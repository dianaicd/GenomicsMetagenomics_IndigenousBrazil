\documentclass[]{article}
\usepackage{lmodern}
\usepackage{amssymb,amsmath}
\usepackage{ifxetex,ifluatex}
\usepackage{fixltx2e} % provides \textsubscript
\ifnum 0\ifxetex 1\fi\ifluatex 1\fi=0 % if pdftex
  \usepackage[T1]{fontenc}
  \usepackage[utf8]{inputenc}
\else % if luatex or xelatex
  \ifxetex
    \usepackage{mathspec}
  \else
    \usepackage{fontspec}
  \fi
  \defaultfontfeatures{Ligatures=TeX,Scale=MatchLowercase}
\fi
% use upquote if available, for straight quotes in verbatim environments
\IfFileExists{upquote.sty}{\usepackage{upquote}}{}
% use microtype if available
\IfFileExists{microtype.sty}{%
\usepackage{microtype}
\UseMicrotypeSet[protrusion]{basicmath} % disable protrusion for tt fonts
}{}
\usepackage[margin=1in]{geometry}
\usepackage{hyperref}
\hypersetup{unicode=true,
            pdftitle={Isotopic analysis},
            pdfauthor={Cruz-Dávalos, Diana I.},
            pdfborder={0 0 0},
            breaklinks=true}
\urlstyle{same}  % don't use monospace font for urls
\usepackage{longtable,booktabs}
\usepackage{graphicx,grffile}
\makeatletter
\def\maxwidth{\ifdim\Gin@nat@width>\linewidth\linewidth\else\Gin@nat@width\fi}
\def\maxheight{\ifdim\Gin@nat@height>\textheight\textheight\else\Gin@nat@height\fi}
\makeatother
% Scale images if necessary, so that they will not overflow the page
% margins by default, and it is still possible to overwrite the defaults
% using explicit options in \includegraphics[width, height, ...]{}
\setkeys{Gin}{width=\maxwidth,height=\maxheight,keepaspectratio}
\IfFileExists{parskip.sty}{%
\usepackage{parskip}
}{% else
\setlength{\parindent}{0pt}
\setlength{\parskip}{6pt plus 2pt minus 1pt}
}
\setlength{\emergencystretch}{3em}  % prevent overfull lines
\providecommand{\tightlist}{%
  \setlength{\itemsep}{0pt}\setlength{\parskip}{0pt}}
\setcounter{secnumdepth}{0}
% Redefines (sub)paragraphs to behave more like sections
\ifx\paragraph\undefined\else
\let\oldparagraph\paragraph
\renewcommand{\paragraph}[1]{\oldparagraph{#1}\mbox{}}
\fi
\ifx\subparagraph\undefined\else
\let\oldsubparagraph\subparagraph
\renewcommand{\subparagraph}[1]{\oldsubparagraph{#1}\mbox{}}
\fi

%%% Use protect on footnotes to avoid problems with footnotes in titles
\let\rmarkdownfootnote\footnote%
\def\footnote{\protect\rmarkdownfootnote}

%%% Change title format to be more compact
\usepackage{titling}

% Create subtitle command for use in maketitle
\newcommand{\subtitle}[1]{
  \posttitle{
    \begin{center}\large#1\end{center}
    }
}

\setlength{\droptitle}{-2em}

  \title{Isotopic analysis}
    \pretitle{\vspace{\droptitle}\centering\huge}
  \posttitle{\par}
    \author{Cruz-Dávalos, Diana I.}
    \preauthor{\centering\large\emph}
  \postauthor{\par}
      \predate{\centering\large\emph}
  \postdate{\par}
    \date{4/3/2018}


\begin{document}
\maketitle

\hypertarget{radiocarbon-measurements}{%
\subsection{Radiocarbon measurements}\label{radiocarbon-measurements}}

\begin{figure}
\includegraphics{isotopes_files/figure-latex/age-1} \caption{\label{fig:age} Figure 1. Uncalibrated dates. The bars represent one standard deviation.}\label{fig:age}
\end{figure}

21 individuals were dated at \textsuperscript{14}CHRONO Center at
Queen's University Belfast. The dates are shown in Table 1 and in Figure
1, along with the dates for MN00015 (Bot15) and MN00017 (Bot17)
presented in Malaspinas et al.~(2014). The uncalibrated dates are
between 23 BP and 327 BP for the Botocudo individuals, and 1080 BP and
2035 BP for the Native American and the Sambaqui individual,
respectively.

\begin{longtable}[]{@{}llrrllrrr@{}}
\caption{Table 1. Isotope values and uncalibrated dates.}\tabularnewline
\toprule
ID & Material type & Age (BP) & SD & Details & Endogenous DNA & d13C &
d15N & C:N ratio\tabularnewline
\midrule
\endfirsthead
\toprule
ID & Material type & Age (BP) & SD & Details & Endogenous DNA & d13C &
d15N & C:N ratio\tabularnewline
\midrule
\endhead
MN00119 & tooth, 1st lower right premolar & 23 & 22 & Botocudo\_Doce &
9.92\% & -18.3 & 12.1 & 3.14\tabularnewline
MN00316 & tooth, 1st upper right molar, child (broken) & 40 & 30 &
Botocudo\_Aimores & 1.33\% & -19.5 & 12.6 & 3.15\tabularnewline
MN00118 & tooth, 3rd upper right molar & 48 & 33 & Botocudo\_Doce &
10.12\% & -19.2 & 11.0 & 3.15\tabularnewline
MN00021 & tooth, 1st upper right molar & 66 & 50 & Botocudo\_Mutum &
9.97\% & -18.7 & 13.1 & 3.19\tabularnewline
MN00346 & tooth, 2nd lower right premolar & 76 & 30 &
Botocudo\_Unknown(Botocudo) & 8.34\% & -19.5 & 13.0 &
3.14\tabularnewline
MN00069 & tooth, upper right canine & 79 & 39 & Botocudo\_Mutum & 0.99\%
& -19.2 & 14.0 & 3.15\tabularnewline
MN00010 & tooth, 1st upper right molar & 84 & 22 & Botocudo\_Itamacuari
& 0.15\% & -15.4 & 12.8 & 3.16\tabularnewline
MN00039 & tooth, 2nd upper right premolar & 106 & 30 & Botocudo\_Mutum &
0.42\% & -17.2 & 11.6 & 3.16\tabularnewline
MN00068 & tooth, 2nd upper right molar & 121 & 33 & Botocudo\_Mutum &
3.33\% & -19.3 & 11.7 & 3.16\tabularnewline
MN00023 & tooth, 2nd upper left premolar & 137 & 25 &
Botocudo\_Poxixa\_Mutum & 0.37\% & -18.3 & 14.5 & 3.19\tabularnewline
MN00022 & teeth (2), 1st upper right molar(deciduous and permanent) &
145 & 30 & Botocudo\_Mutum & 2.15\% & -14.3 & 15.1 & 3.16\tabularnewline
MN00013 & tooth, 1st lower right molar & 150 & 40 & Botocudo\_Mutum &
17.43\% & -18.1 & 12.8 & 3.17\tabularnewline
MN0003 & tooth, 1st upper left molar & 160 & 26 & Botocudo\_Mutum &
5.30\% & -19.0 & 12.3 & 3.17\tabularnewline
MN00056 & tooth, 3rd lower right molar & 169 & 25 & Botocudo\_Pot &
35.23\% & - & 19.6 & 13.5 3.15\tabularnewline
MN00067 & tooth, 2nd upper left molar & 170 & 21 & Botocudo\_Minas &
6.33\% & -16.4 & 12.5 & 3.16\tabularnewline
MN0009 & tooth, 2nd upper righ molar & 185 & 29 & Botocudo\_Mucuri &
17.55\% & -19.4 & 13.2 & 3.16\tabularnewline
MN00064 & tooth, 2nd upper right premolar & 199 & 41 & Botocudo\_Doce &
4.74\% & -19.7 & 11.5 & 3.14\tabularnewline
MN00045 & tooth, 1st upper left molar (child) & 249 & 29 &
Botocudo\_Mutum & 3.17\% & -15.5 & 15.3 & 3.16\tabularnewline
MN00016 & tooth, 2nd upper right molar & 327 & 24 & Botocudo\_Itapemirim
& 3.93\% & -19.5 & 12.8 & 3.16\tabularnewline
Bot15 & Bot15 & 417 & 25 & Bot15 & & -17.0 & 13.9 & 3.20\tabularnewline
Bot17 & Bot17 & 487 & 25 & Bot17 & & -14.8 & 18.2 & 3.20\tabularnewline
MN1943 & tooth, 3rd lower right molar & 1080 & 25 &
NativeAmerican\_MorrodaBabilonia & 0.82\% & -12.1 & 27.5 &
3.17\tabularnewline
MN01701 & tooth, 2nd lower left molar & 2035 & 27 & Sambaqui\_Cabecuda &
0.09\% & -10.3 & 17.5 & 3.16\tabularnewline
\bottomrule
\end{longtable}

\hypertarget{delta13c-and-delta15n-measurements}{%
\subsection{\texorpdfstring{\(\delta\)\textsuperscript{13}C and
\(\delta\)\textsuperscript{15}N
measurements}{\textbackslash delta13C and \textbackslash delta15N measurements}}\label{delta13c-and-delta15n-measurements}}

We measured the \(\delta\)\textsuperscript{13}C and
\(\delta\)\textsuperscript{15}N values for the 21 individuals. These
values can give us an insight into the dietary habits for a given
organism. However, we need to be careful when examining children's
teeth. The decidious teeth (`baby teeth') develop during the embryonic
stage of development, and the isotopic measurements reflect the dietary
habits of the child's mother. Moreover, the crown of the first molars
are formed during the early years/breastfeeding period. In some
instances, a child's first molar can have enriched \textsuperscript{15}N
values due to breastfeeding, indicating that a significant proportion of
the dentine was formed at that time (Schroeder et al. 2009). In order to
know more about an individual's dietary habits, it is better to date
second and third molars.

In Figure 2, we show whether the samples analyzed belonged to a child,
and whether the tooth was a first molar. Among the samples, we observe
two children's first molars (from MN00022 and MN00045) that could have
shifted \(\delta\)\textsuperscript{15}N values due to breastfeeding. We
have to keep in mind this information for the next steps, in which we
will infer dietary proportions.

\begin{figure}
\includegraphics{isotopes_files/figure-latex/unnamed-chunk-3-1} \caption{\label{fig:isotope} Figure 2. Isotope values. The triangles represent children and the circles represent adults. The color indicates whether we examined a first molar (purple) or any other type of tooth (yellow).}\label{fig:unnamed-chunk-3}
\end{figure}

\hypertarget{mixing-models-for-stable-isotopic-data}{%
\subsection{Mixing models for stable isotopic
data}\label{mixing-models-for-stable-isotopic-data}}

We used SIAR (Stable Isotope Analysis in R, ({\textbf{???}})) in order
\textbf{to infer dietary proportions} for the individuals.

\hypertarget{model}{%
\subsubsection{Model}\label{model}}

SIAR fits a model via Markov chain Monte Carlo. The parameters to fit
are the \textbf{dietary proportions}.

Keypoints of the model:

\begin{itemize}
\tightlist
\item
  a consumer (e.g., a Native American) integrates isotopes from one or
  more sources (e.g., marine mammals, marine fish, terrestrial fauna,
  etc.)
\item
  we have known values of the trophic enrichment factors (TEF), which
  indicate the difference in the isotopic ratio between a consumer and
  their diet
\item
  a consumer incorporates an isotope proportionally to the dietary
  proportion of a given source, its TEF and its isotope value
\item
  for a consumer, the observed isotope value is explained by the sum of
  the dietary proportions
\end{itemize}

We used the default Dirichlet prior distribution for the parameters.
This distribution allows us to treat source data independently; e.g.,
the estimated dietary proportion for source X does not depend on the
estimated proportion for source Y. All inferred dietary proportions must
sum to one.

We fitted a model per individual. As suggested in the manual, we set
500,000 iterations and discarded the first 50,000 iterations.

\hypertarget{results}{%
\subsubsection{Results}\label{results}}

\hypertarget{raw-data}{%
\paragraph{Raw data}\label{raw-data}}

\begin{figure}
\includegraphics{isotopes_files/figure-latex/unnamed-chunk-5-1} \caption{\label{fig:rawdata}Figure 3. The squares represent the mean isotope values for a consumer whose diet was
based on a specific source (terrestrial fauna, marine fish or marine mammals).
The dashed lines represent two standard errors of the mean isotope values.
The solid circles represent the pair of isotope values reported by the CHRONO Centre per individual. 
The triangles correspond to the isotope values for Bot15 and Bot17 (from Malaspinas et al., 2014),
dated at the AMS 14C Centre at Aarhus University}\label{fig:unnamed-chunk-5}
\end{figure}

We had a first inspection of the isotope values for the 23 individuals
(21 dated in this study plus Bot15 and Bot17), along with the expected
isotope values for an organism with a diet based on terrestrial fauna,
marine fish or marine mammals (Figure 3). In order to obtain the
coordinates of a (theoretical) consumer whose diet consisted on a single
source, we summed the mean TEF values and the mean isotope values of a
source.

We observe that for the Cabeçuda individual (or Sambaqui) the isotope
values correspond to a marine diet, in agreement with the origin of the
individual. The Native American from Morro da Babilonia shows an unusual
high value for \(\delta\)\textsuperscript{15}N, and we need to contact
the \textsuperscript{14}CHRONO Centre to get more information. Regarding
the Botocudo individuals, most of the samples group close to the
expected values for a diet based on terrestrial fauna. Interestingly,
the Polynesian individuals (Bot15 and Bot17) fall on the limits or
outside of the expected isotope values for a marine diet.

\hypertarget{inferred-dietary-proportions}{%
\subsubsection{Inferred dietary
proportions}\label{inferred-dietary-proportions}}

The inferred dietary proportions for the 23 individuals are listed in
Table S1, and displayed on in Figures 4-6. We noticed that, although
MN00045 and MN00022 are among the individuals from the Botocudo
collection with highest proportions of marine diets, this might be due
to the type of material that was sent to date (children's first molars).
The program also inferred a high proportion of marine diet for the
Native American (Morro da Babilônia) due to the anormal
\(\delta\)\textsuperscript{15}N value obtained for this individual.

\hypertarget{boxplots}{%
\paragraph{Boxplots}\label{boxplots}}

\begin{figure}
\centering
\includegraphics{All_proportions.png}
\caption{Figure 4. Dietary proportion (boxplots). The samples from
children's first molars are indicated with a '*' in front of the
individual's ID.}
\end{figure}

\hypertarget{densities}{%
\paragraph{Densities}\label{densities}}

\begin{figure}
\centering
\includegraphics{All_densities.png}
\caption{Figure 5. Dietary proportion (densities). The samples from
children's first molars are indicated with a '*' in front of the
individual's ID.}
\end{figure}

\hypertarget{diagnostic-matrix-plot}{%
\paragraph{Diagnostic matrix plot}\label{diagnostic-matrix-plot}}

High correlations indicate that the program has problems to distinguish
between the dietary proportions of two sources. Thus, as expected, the
draws for marine fish and marine mammals tend to be highly correlated,
as their isotope values are similar.

\begin{figure}
\centering
\includegraphics{All_matrix.png}
\caption{Figure 6. Dietary proportions (histograms and correlations).
For every individual, we have a matrix with nine entries: The histograms
of the estimated proportions are located on the diagonal; on the upper
entries, the contour plots depict whether the posterior distributions
are correlated; the lower entries contain the values corresponding to
the correlation coefficient for pairs of distribution. The samples from
children's first molars are indicated with a '*' in front of the
individual's ID.}
\end{figure}

\hypertarget{traces}{%
\paragraph{Traces}\label{traces}}

For every individual, we saved the sampled values of 30,000 out of
500,000 draws for every estimated dietary proportion after removing the
burnin phase (first 50,000 iterations).

\begin{figure}
\centering
\includegraphics{All_trace_14July2018.png}
\caption{Figure 7. MCMC traces (after burnin phase). A point represents
a sampled value at a given iteration. The samples from children's first
molars are indicated with a '*' in front of the individual's ID.}
\end{figure}

\hypertarget{calibrated-dates}{%
\subsection{Calibrated dates}\label{calibrated-dates}}

Finally, we calibrated the radiocarbon results using a Southern
Hemisphere calibration curve (ShCal13, Reimer et al. (2013)) for all
individuals. The estimated dietary proportions are listed in Table S1.

Additionally, for Bot15, Bot17 and MN01701 (Cabeçuda from Sambaqui), we
corrected for a marine reservoir effect, according to the estimated
carbon uptake from a marine diet, and added a local \(\Delta\)R offset
of 33±24 years, as suggested during our visit to the Museu Macional. The
parameters specific to the individuals with marine diets are presented
in Table 2, along with the parameters reported in Malaspinas et
al.~(2014) for Bot15 and Bot17. Notably, we obtained higher proportions
of marine diet for Bot15 and Bot17 than those reported in Malaspinas et
al. (2014) (Table 2).

\begin{longtable}[]{@{}lcc@{}}
\caption{Table 2. Calibration parameters for individuals with possible
marine diets.}\tabularnewline
\toprule
& Malaspinas et al.~(2014) & This study\tabularnewline
\midrule
\endfirsthead
\toprule
& Malaspinas et al.~(2014) & This study\tabularnewline
\midrule
\endhead
Marine carbon protein (Bot15) & 30±16\% & 39±18\%\tabularnewline
Marine carbon protein (Bot17) & 60±16\% & 63±18\%\tabularnewline
Marine carbon protein (MN01701) & NA & 89±14\%\tabularnewline
Atmospheric calibration curve & ShCal04 (Board 2004) & ShCal13 (Reimer
et al. 2013)\tabularnewline
Marine calibration curve & Marine09 (Reimer et al. 2009) & Marine13
(Reimer et al. 2013)\tabularnewline
\(\Delta\)R offset & 0±20 & 33±24\tabularnewline
\bottomrule
\end{longtable}

We used the OxCal programme, (version 4.3, Bronk Ramsey (2009)) to
calibrate the radiocarbon results. The figures below represent the
posterior density for the \textsuperscript{14}C dates. We show in
parentheses the date and standard deviation reported by
\textsuperscript{14}CHRONO Centre, as well as the agreement indices
(``A'') and convergence integral (``C''") returned by OxCal (Figure 8).
The agreement indices are a measure of the agreement between the model
(prior) and the observational data (likelihood); this value should be
over 60\% (although in some instances we recovered values of 101\%). The
convergence integral is a test of the effectiveness of the MCMC
algorithm; this value should be above 95\%.

Regarding Bot15 and Bot17, the probability density are in agreement with
the curves presented in Malaspinas et al. (2014) (Figure 8B). Finally,
the calibrated dates for the Botocudo individuals indicate that they are
post-Columbian (Figure 8C).

\begin{figure}
\centering
\includegraphics{All_calibrated.png}
\caption{Figure 8. Calibrated \textsuperscript{14}C dates for (A) MN1943
and MN01701, (B) Bot15 and Bot17 from Malaspinas et al.~(2014) and from
this study and (C) the Botocudo collection (this study only). The
numbers in parentheses refer to the uncalibrated dates; in brackets, we
report the agreement indices and the convergence integral values. The
circles indicate the estimated mean, and the bars represent one standard
deviation. The horizontal brackets denote the regions associated to the
95\% highest posterior density.}
\end{figure}

\hypertarget{references}{%
\subsection*{References}\label{references}}
\addcontentsline{toc}{subsection}{References}

\hypertarget{refs}{}
\leavevmode\hypertarget{ref-Board2004}{}%
Board, Arizona. 2004. ``2004 by the Arizona Board of Regents on Behalf
of the University of Arizona.'' \emph{Radiocarbon} 46 (1): 1111--50.
\url{https://doi.org/10.2458/azu_js_rc.46.4183}.

\leavevmode\hypertarget{ref-BronkRamsey2009}{}%
Bronk Ramsey, Christopher. 2009. ``Bayesian Analysis of Radiocarbon
Dates.'' \emph{Radiocarbon} 51 (01): 337--60.
\url{https://doi.org/10.1017/S0033822200033865}.

\leavevmode\hypertarget{ref-Malaspinas2014a}{}%
Malaspinas, A.-S., O. Lao, H. Schroeder, M. Rasmussen, M. Raghavan, I.
Moltke, P. F. Campos, et al. 2014. ``Two Ancient Human Genomes Reveal
Polynesian Ancestry Among the Indigenous Botocudos of Brazil.''
\emph{Current Biology} 29 (8): 12711--6.
\url{https://doi.org/10.1016/j.cub.2014.09.035}.

\leavevmode\hypertarget{ref-Reimer2013}{}%
Reimer, Paula J., Edouard Bard, Alex Bayliss, J. Warren Beck, Paul G.
Blackwell, Christopher Bronk Ramsey, Caitlin E. Buck, et al. 2013.
``IntCal13 and Marine13 Radiocarbon Age Calibration Curves 0--50,000
Years Cal BP.'' \emph{Radiocarbon} 55 (04): 1869--87.
\url{https://doi.org/10.2458/azu_js_rc.55.16947}.

\leavevmode\hypertarget{ref-Reimer2009}{}%
Reimer, P. J., M. G. L. Baillie, E. Bard, a. Bayliss, J. W. Beck, P. G.
Blackwell, C. Bronk Ramsey, et al. 2009. ``INTCAL 09 and MARINE09
Aadiocarbon Age Calibration Curves, 0-50,000 Years Cal BP.''
\emph{Radiocarbon} 51 (4): 1111--50.
\url{https://doi.org/10.2458/azu_js_rc.51.3569}.

\leavevmode\hypertarget{ref-Schroeder2009}{}%
Schroeder, Hannes, Tamsin C. O'Connell, Jane A. Evans, Kristrina A.
Shuler, and Robert E. M. Hedges. 2009. ``Trans-Atlantic Slavery:
Isotopic Evidence for Forced Migration to Barbados.'' \emph{American
Journal of Physical Anthropology} 139 (4): 547--57.
\url{https://doi.org/10.1002/ajpa.21019}.


\end{document}
